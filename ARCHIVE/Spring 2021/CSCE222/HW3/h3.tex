\documentclass[journal]{IEEEtran}

\usepackage{blindtext}
\usepackage{cite}
\usepackage{graphicx}
\usepackage{array}
\usepackage{color}
\usepackage{tabularx}
\usepackage{epsfig}
\usepackage{amsmath}
\usepackage{amssymb}
\usepackage{bm}
\usepackage{wasysym}
\usepackage{circuitikz}
\usepackage{float}
\usetikzlibrary{arrows,shapes,calc,positioning}

\newcommand{\myscope}[2]
{\draw[thick,rotate=#2] (#1) circle (12pt)
(#1) ++(-0.35,-0.1) --++ (0.3,0.3) --++ (0,-0.3) --++(0.3,0.3) --++(0,-0.3);}

\begin{document}

\title{CSCE 222 \\ Homework 3}

\author{Jacob~Purcell,~\IEEEmember{Texas~A\&M,~Student}}

\maketitle
\section*{2.3}
\subsection*{12}
    For a function to be one-one, the following condition must hold;
    \begin{equation}
    f(n_1) = f(n_2)
    \end{equation}
    $$\therefore n_1 = n_2$$
    \subsubsection*{a)}
        \begin{equation}
            f(n) = n - 1
        \end{equation}
        \begin{equation}
            n_1 - 1 = n_2 - 1
        \end{equation}
        
        Add 1 to both sizes;
        \begin{equation}
            n_1 = n_2
        \end{equation}
        $$\boxed{\therefore one-one}$$

    \subsubsection*{b)}
        \begin{equation}
            f(n) = n^2 + 1
        \end{equation}
        \begin{equation}
            n_1^2 + 1 = n_2^2 + 1
        \end{equation}

        Subtract 1 from both sides;
        \begin{equation}
            n_1^2 = n_2^2
        \end{equation}

        Square root both sides;
        \begin{equation}
            \pm n_1 = \pm n_2
        \end{equation}

        This creates the situation that 
        \begin{equation}
            n_1 = \pm n_2
        \end{equation}

        Where two inputs are mapped to one output,
        \begin{equation}
            \boxed{\therefore not~one-one}
        \end{equation}

    \subsubsection*{c)}
        \begin{equation}
            f(n) = n^3
        \end{equation}
        \begin{equation}
            n_1^3 = n_2^3
        \end{equation}
        Take the cube root,
        \begin{equation}
            n_1 = n_2
        \end{equation}
        \begin{equation}
            \boxed{\therefore one-one}
        \end{equation}

    \subsubsection*{d)}
        \begin{equation}
            f(n) = \frac{n}{2}
        \end{equation}
        \begin{equation}
            \frac{n_1}{2} = \frac{n_2}{2}
        \end{equation}
        Multiply each side by 2,
        \begin{equation}
            \boxed{\therefore one-one}
        \end{equation}

\subsection*{14}
        To check whether the functions are surjective, an inverse will be found 
        for each function, then plugged back into the original equation to verify 
        freedom from contradiction. The condition that $f(m,n) = y$ is understood.
    \subsubsection*{a)}
        \begin{equation}
            y = 2m - n
        \end{equation}
        Solve for $n$,
        \begin{equation}
            n = 2m - y
        \end{equation}
        Plug in $n$,
        \begin{equation}
            y = 2m - (2m - y)
        \end{equation}
        Remove parenthases and distribute negation,
        \begin{equation}
            y = 2m - 2m + y
        \end{equation}
        Add like terms
        \begin{equation}
            y = y
        \end{equation}
        \begin{equation}
            \boxed{\therefore surjective}
        \end{equation}

    \subsubsection*{b)}
        \begin{equation}
            y = m^2 - n^2
        \end{equation}
        Solve for $n$,
        \begin{equation}
            n = \pm \sqrt{m^2 - y}
        \end{equation}
        From here, it is apparent that part of the domain lies within  $\mathbb{I}$ outside of $\mathbb{Z}$ in 
        the case that $y > m^2$.
        \begin{equation}
            \boxed{\therefore not~surjective}
        \end{equation}

    \subsubsection*{c)}
        \begin{equation}
            y = m + n + 1
        \end{equation}
        Solve for $n$,
        \begin{equation}
            n = y - m - 1
        \end{equation}
        Plug in $n$,
        \begin{equation}
            y = m + y - m - 1 + 1
        \end{equation}
        Combine like terms,
        \begin{equation}
            y = y
        \end{equation}
        \begin{equation}
            \boxed{\therefore surjective}
        \end{equation}

    \subsubsection*{d)}
        \begin{equation}
            y = \mid m \mid - \mid n \mid
        \end{equation}
        This creates the situation that $(m > n) \exists \mathbb{Z}^+$,
        \begin{equation}
            y^+ = m - n
        \end{equation}
        or $(m < n) \exists \mathbb{Z}^-$
        \begin{equation}
            y^- = n - m
        \end{equation}
        Solving for $n$,
        \begin{equation}
            n = m-y^+,~ \\
            n = m+y^-
        \end{equation}
        Plug in $n$,
        \begin{equation}
            y^+ = m - m + y^+,~ \\
            y^- = m + y^- - m
        \end{equation}
        Combine like terms,
        \begin{equation}
            y^+ = y^+,~ \\
            y^- = y^-
        \end{equation}

        Where $y^+ \exists \mathbb{Z}^+$ and $y^- \exists \mathbb{Z}^-$. The solution $f(0,0) = 0$ is understood.
        This creates a situation where $y$ can be any element of $\mathbb{Z}$. 

        \begin{equation}
            \boxed{\therefore surjective}
        \end{equation}

    \subsubsection*{e)}
        \begin{equation}
            y = m^2 - 4
        \end{equation}
        Solve for m,
        \begin{equation}
            m = \pm \sqrt{y+4}
        \end{equation}
        If $y < -4, m\exists \mathbb{I}$ which is outside of the domain of $\mathbb{Z}$.
        \begin{equation}
            \boxed{\therefore not~surjective}
        \end{equation}

\subsection*{20}
    \subsubsection*{a)}
        Injective but not surjective,
        \begin{equation}
            \boxed{f(x) = \frac{1}{x}}
        \end{equation}
        Every input has one output, however, $y=0$ is not mapped by an input.

    \subsubsection*{b)}
    Surjective not injective,
    \begin{equation}
        \boxed{f(x,z) = x^2 - z^2}
    \end{equation}
    This covers the entire domain from $N-N$ however certain values are mapped to multiple times($e.g.~x = \pm N, z = 0$).
    
    \subsubsection*{c)}
    Bijective,
    \begin{equation}
        \boxed{f(x) = x}
    \end{equation}
    In this case, every output corresponds to one output, and every output is mapped.

    \subsubsection*{d)}
    General function,
    \begin{equation}
        \boxed{f(x) = x^2}
    \end{equation}
    Two inputs correspond to one output and negative outputs are not covered in the domain $x \exists \mathbb{R}$.

\subsection*{22}
    For bijection, a function must be injective and surjective.
    The following proofs will reuse methods from previous parts.
    \subsubsection*{a)}
    \begin{equation}
        f(x) = -3x + 4
    \end{equation}
        Injective: $f(y) = f(x)$
        \begin{equation}
            -3x + 4 = -3y + 4
        \end{equation}
        After adding $4$ and dividing by $-3$, the expession becomes
        \begin{equation}
            x = y
        \end{equation}
        $$\therefore injective$$

        Surjective: $f(f^{-1}(x)) \exists R, y = f(x)$
        \begin{equation}
            y = -3x + 4
        \end{equation}
        Solve for $x$,
        \begin{equation}
            x = \frac{4-y}{3}
        \end{equation}
        Plug $x$ into original function,
        \begin{equation}
            y = -3(\frac{4-y}{3}) + 4
        \end{equation}
        Distribute 3,
        \begin{equation}
            y = -(4-y) + 4
        \end{equation}
        Distribute negative and combine like terms,
        \begin{equation}
            y = y
        \end{equation}
        $$\therefore surjective$$
        This function is surjective and injective,
        $$\boxed{\therefore bijective}$$

    \subsubsection*{b)}
        \begin{equation}
            f(x) = -3x^2 + 7
        \end{equation}
        Injective: $f(y) = f(x)$
        \begin{equation}
            -3x^2 + 7 = -3y^2 + 7
        \end{equation}
        Subtract $7$ from both sides and divide $-3$,
        \begin{equation}
            x^2 = y^2
        \end{equation}
        take root $2$ of both sides,
        \begin{equation}
            \pm x = \pm y
        \end{equation}
        \begin{equation}
            x = \pm y
        \end{equation}
        Since two inputs map to the same output, this function is not injective.
        $$\boxed{\therefore not~bijective}$$


    \subsubsection*{c)}
        \begin{equation}
            f(x) = \frac{x-1}{x+2}
        \end{equation}
        Surjective: $f(f^{-1}(x)) \exists R, y = f(x)$
        \begin{equation}
            y = \frac{x-1}{x+2}
        \end{equation}
        Solving for $x$, multiply by $x+2$
        \begin{equation}
            (x+2)y = x-1
        \end{equation}
        Distribute $y$,
        \begin{equation}
            xy+2y = x-1
        \end{equation}
        Collect like variables,
        \begin{equation}
            xy-x = -2y-1
        \end{equation}
        Factor $x$, divide $y+1$
        \begin{equation}
            x = \frac{-2y-1}{y-1}
        \end{equation}
        There is a discontinuity at $y = 1$, and therefore does not have an input mapping
        and is not an element of $\mathbb{R}$, making this function not surjective.
        $$\boxed{\therefore not~bijective}$$


    \subsubsection*{d)}
        \begin{equation}
            f(x) = x^5 + 1
        \end{equation}
        Injective: $f(y) = f(x)$
        \begin{equation}
            y^5 + 1 = x^5 + 1
        \end{equation}
        subtract 1, take root $5$,
        \begin{equation}
            y = x
        \end{equation}
        $$\therefore injective$$
        
        Surjective: $f(f^{-1}(x)) \exists R, y = f(x)$
        \begin{equation}
            y = x^5 + 1
        \end{equation}
        Solve for y,
        \begin{equation}
            x = (y-1)^{\frac{1}{5}}
        \end{equation}
        plug x into original function,
        \begin{equation}
            y = ((y-1)^{\frac{1}{5}})^5 + 1
        \end{equation}
        Multiply explonential terms per rules of exponents,
        \begin{equation}
            y = y-1 + 1
        \end{equation}
        Add constants,
        \begin{equation}
            y = y
        \end{equation}
        $$\therefore surjective$$
        Both injective and surjective,
        $$\boxed{\therefore bijective}$$
        
\subsection*{60}
        There are $8$ bits per byte, dimensional analysis says that the total number
        of bits divided by the equivalent corresponding number of bytes
        (as determined by how many bytes are in a bit) is 1, which gives rise to the equation
        \begin{equation}
            \frac{x~bits}{8y~bytes} = 1, ~this~says~that~if~we~have~8~bits,~we~have~1~byte
        \end{equation}
        We want a form that outputs a number of bytes given a number of bits,
        multiplying both sides by number of bytes yields
        \begin{equation}
            x~bits = 8y~bytes
        \end{equation}
        It is noted that $y$ is a function of $x$ and $x$ is a function of $y$,
        which means that we can attain a bit count for any number of bytes or 
        a byte count for any number of bits. For this excersize we will consider only the following case,
        \begin{equation}
            y(x) = \frac{x}{8}
        \end{equation}
        Since bytes are discrete packages of 8 bits, any decimal greater than $0$ will result in a new byte.

        \subsubsection*{a)}
        \begin{equation}
            y(4) = \frac{4}{8}
        \end{equation}
        divide,
        \begin{equation}
            y = \frac{1}{2} = 0.5 \approx \boxed{1}
        \end{equation}

        \subsubsection*{b)}
        \begin{equation}
            y(10) = \frac{10}{8}
        \end{equation}
        divide,
        \begin{equation}
            y = \frac{5}{4} = 1.25 \approx \boxed{2}
        \end{equation}

        \subsubsection*{c)}
        \begin{equation}
            y(500) = \frac{500}{8}
        \end{equation}
        divide,
        \begin{equation}
            y = \frac{125}{2} = 62.5 \approx \boxed{63}
        \end{equation}

        \subsubsection*{d)}
        \begin{equation}
            y(x) = \frac{3000}{8}
        \end{equation}
        divide,
        \begin{equation}
            y = \boxed{375}
        \end{equation}


\end{document}
