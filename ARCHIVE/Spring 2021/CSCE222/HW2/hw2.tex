\documentclass[journal]{IEEEtran}

\usepackage{blindtext}
\usepackage{cite}
\usepackage{graphicx}
\usepackage{array}
\usepackage{color}
\usepackage{tabularx}
\usepackage{epsfig}
\usepackage{amsmath}
\usepackage{amssymb}
\usepackage{bm}
\usepackage{wasysym}
\usepackage{circuitikz}
\usepackage{float}
\usetikzlibrary{arrows,shapes,calc,positioning}

\newcommand{\myscope}[2]
{\draw[thick,rotate=#2] (#1) circle (12pt)
(#1) ++(-0.35,-0.1) --++ (0.3,0.3) --++ (0,-0.3) --++(0.3,0.3) --++(0,-0.3);}

\begin{document}

\title{CSCE~222 \\ Homework 2}

\author{Jacob~Purcell,~\IEEEmember{Texas~A\&M,~Student}}

\maketitle


\section*{1.6}
	\subsection*{4}
		\subsubsection{a}
		Kangaroos($p$) live in($\rightarrow$) Australia($q$) and($\cdot$) are marsupials($r$). \\
		Therefore, kangaroos($p$) are($\rightarrow$) marsupials($r$). $\boxed{Hypothetical Syllogism}$

		\subsubsection{b}
		It is either hotter than 100 degrees today($p$) or($+$) the pollution is dangerous($q$). \\
		It is less than 100 degrees outside today($p'$). \\
		Therefore, the pollution is dangerous($\therefore q$). $\boxed{Disjunctive Syllogism}$

		\subsubsection{c}
		Linda is an excellent swimmer($p$). \\
		If Linda is an excellent swimmer($p$), then($\rightarrow$) she can work as a lifeguard($q$).\\
		Therefore, Linda can work as a lifeguard($\therefore q$). $\boxed{Modus Ponens}$

		\subsubsection{d}
		Steve will work at a computer company this summer($p$).\\
		Therefore, this summer Steve will work at a computer
		company or he will be a beach bum($\therefore p+q$). $\boxed{Addition}$

		\subsubsection{e}
		If I work all night on this homework($p$), then($\rightarrow$) I can answer all the exercises($q$). \\
		If I answer all the exercises($q$),($\rightarrow$) I will understand the material($r$). \\
		Therefore, if I work all night on this homework, then I will understand the
		material($\boxed{p \rightarrow r}$). $\boxed{Hypothetical Syllogism}$

	\subsection*{10}
		\subsubsection{a}
		 If I play hockey($p$), then($\rightarrow$) I am sore the next day($q$).  \\
		 I use the whirlpool($r$) if($\leftarrow$) I am sore($q$).  \\
		 I did not use the whirlpool($r'$).  \\ $\boxed{\therefore p'}$ \\
		 Using Hypothetical Syllogism the expression becomes $p \rightarrow r$.
		 The boxed conclusion was drawn using Modus Tollens.

		\subsubsection{b}
		 If I work($p$), it($\rightarrow$) is either sunny($q$) or($+$) partly sunny($r$). \\
		 I worked($p$) ($\rightarrow$)last Monday($s$) or($+$) I worked last Friday($t$).   \\
		 It was not sunny($q'$) on($\cdot$) Tuesday($u$).   \\
		 It was not partly($r'$) sunny on Friday($t$).\\ $\boxed{\therefore s \cdot q}$ \\
		 Every combination evaluates to not working except if it is sunny on monday.
		 Through simplification, it is deduced that since $p$ is true, $\therefore sq \equiv T$.

		\subsubsection{c}
		 All insects($o$) have($\equiv$) six legs($p$).   \\
		 Dragonflies($m$) are($\rightarrow$) insects($o$). \\
		 Spiders($l$) do not ($\rightarrow$)have six legs($p'$).   \\
		 Spiders($l$) eat($\rightarrow$) dragonflies($m'$). \\
		 Line 3 says $l \rightarrow p'o'$, $\boxed{\therefore l \equiv m'}$

		\subsubsection{d}
		Every student($p$) has($\rightarrow$) an Internet account($q$). \\
		Homer($r$) does($\rightarrow$) not have an Internet account($q'$). \\
		Maggie($s$) has($\rightarrow$) an Internet account($q$). \\
		Using Disjunctive Syllogism, we can conclude that homer is not a student, $\therefore$ maggie is a student.

	\subsection*{16}
		\subsubsection{a}
		Everyone enrolled in the university($T \equiv$) has lived in a dormitory($p$). \\
		Mia($q$) has($\rightarrow$) never lived in a dormitory($p'$). \\
		Therefore, Mia is not enrolled in the university($\therefore q'$). \\
		This statement is $\boxed{true}$ since line 2 implies that a tautology is false if $q$ is true.

		\subsubsection{b}
		A convertible car($p$) is($\rightarrow$) fun to drive($q$). \\
		Isaac’s car is not a convertible($p'$). \\
		Therefore, Isaac’s car is not fun to drive($\therefore q'$).\\
		This statement is $\boxed{false}$ since $p' \rightarrow q$ evaluates to 
		true or false when line 3 states that it evaluates to false. Another way to think of it would be 
		that it is never declared that convertibles are the only fun car.

\section*{1.7}
	\subsection*{16}
		In the case that $x,y,x$ are all an even number($n$),
		$$3*n\%2=0$$
		The only possibility left is that at least one of $x,y,$ or $z$ is odd.

	\subsection*{18}
		If $m$ and $n$ are odd, 
		$$m\%2=1, n\%2=1, (m\%2)(n\%2)=1, \therefore mn\%2 = 1$$
		In order for $mn\%2=0$, $m\%2 = 0$, or $n\%2=0$.

	\subsection*{20}
	\subsubsection{a}
	Suppose $n$ is odd, 
	$$3n\%2 = 1, 2\%2=0, \therefore (3n+2)\%2=1$$

	\subsubsection{b}
	Suppose $(3n+2)\%2=1$ and $n\%2=0$. $3n\%2$ must be $1$, however, $3n\%2 = 0$, $\therefore (3n+2)\%2=0$.

	\subsection*{28}
	$$(7n+4)\%2=0$$
	$$(7n\%2)+(4\%2)=0$$
	$$(7\%2)(n\%2)+0=0$$
	$$(1)(n\%2)=0$$
	$$\therefore n\%2=0$$

\section*{2.1}
	\subsection*{46}
	\subsubsection{a}
	There exists a number $x$ on the set of real numbers such that $x^3 = -1$. True since $(-1)^{1/3} = -1$ and $-1 \epsilon \mathbb{R}$.

	\subsubsection{b}
	There exists a number $x$ on the set of integers such that $x+1 > x$. True since $1$ is a positive integer, 
	increasing the observed value $\forall x$.

	\subsubsection{c}
	For all numbers $x$ on the set of integers, $x$ minus 1 is also on the set of integers. True since 
	$x$ and $1$ are on the set of integers, and they are performing a $\mathbb{Z}$ operation that is defined to return
	a value on the same set(subtraction).

	\subsubsection{d}
	For all numbers $x$ on the set of integers, $x^3$ is also on the set of integers. True since 
	$x$ is on the set of integers, and it is performing a $\mathbb{Z}$ operation that is defined to return
	a value on the same set(multiplication).

	\subsection*{48}
		$$x \epsilon \mathbb{Z}$$
	\subsubsection{a}
	$x^3 \ge 1, \boxed{T} \forall x > 0$.

	\subsubsection{b}
	$x^2 = 2, \boxed{F}, 0^2 = 0, 1^2 = 1, 2^2 = 4$. 

	\subsubsection{c}
	$x < x^2, \boxed{T} \forall x > 1$.

\section*{2.2}
	\subsection*{2}
	\subsubsection{a}
	The set of sophomores($A$) taking($\cap$) discrete mathematics in your school($B$).
	$\boxed{A \cap B}$

	\subsubsection{b}
	The set of sophomores($A$) at your school who are not($-$) taking discrete mathematics($B$).
	$\boxed{A-B}$

	\subsubsection{c}
	The set of students at your school who either are sophomores($A$) or($+$) are taking discrete mathematics($B$).
	$\boxed{A+B}$

	\subsubsection{d}
	The set of students at your school who either($+$) are not sophomores($A'$) or are not taking discrete mathematics($B'$)
	$\boxed{A'+B'}$


\end{document}
