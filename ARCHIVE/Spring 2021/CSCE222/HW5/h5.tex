\documentclass[journal]{IEEEtran}

\usepackage{blindtext}
\usepackage{cite}
\usepackage{graphicx}
\usepackage{array}
\usepackage{color}
\usepackage{tabularx}
\usepackage{epsfig}
\usepackage{amsmath}
\usepackage{amssymb}
\usepackage{bm}
\usepackage{wasysym}
\usepackage{circuitikz}
\usepackage{float}
\usetikzlibrary{arrows,shapes,calc,positioning}

\newcommand{\myscope}[2]
{\draw[thick,rotate=#2] (#1) circle (12pt)
(#1) ++(-0.35,-0.1) --++ (0.3,0.3) --++ (0,-0.3) --++(0.3,0.3) --++(0,-0.3);}

\begin{document}

\title{CSCE 222 \\ Homework 5}

\author{Jacob~Purcell,~\IEEEmember{Texas~A\&M,~Student}}

\maketitle
\section*{2.4}
\subsection*{12}
\begin{equation}
    a_n = -3a_{n-1}+4a_{n-2}
\end{equation}
\subsubsection*{a)}
$$a_n = 0$$
$$a_{n-1} = 0$$
$$a_{n-2} = 0$$

plugging in, the expression becomes

$$a_n = 0 + 0 = 0$$
$\boxed{\therefore~is~a~solution}$

\subsubsection*{b)}
$$a_n = 1$$
$$a_{n-1} = 1$$
$$a_{n-2} = 1$$

$$a_n = -3 + 4 = 1$$
$\boxed{\therefore~is~a~solution}$

\subsubsection*{c)}
$$a_n = (-4)^n$$
$$a_{n-1} = (-4)^{n-1}$$
$$a_{n-2} = (-4)^{n-2}$$

$$a_n = -3(-4)^{n-1} + 4(-4)^{n-2} $$
$$a_n = (-4)^{n-2}(12 + 4) = 16(-4)^{n-2}$$
$$16(-4)^{n-2} = (-4)(-4)(-4)^{n-2} = (-4)^{n-2}$$
$\boxed{\therefore~is~a~solution}$

\subsubsection*{d)}
$$a_n = 2(-4)^n+3$$
$$a_{n-1} = 2(-4)^{n-1}+3$$
$$a_{n-2} = 2(-4)^{n-2}+3$$
$$a_n = -6(-4)^{n-1} - 9 + 8(-4)^{n-2} + 12 $$
$$a_n = (-4)^{n-2}(-6 + 8) - 3 = 2(-4)^n+3$$
$\boxed{\therefore~is~a~solution}$


\subsection*{14}
\subsubsection*{a)}
$$a_n= 3$$
$$a_n - a_{n-1} = 3 - 3$$
$$\boxed{a_n = a_{n-1}, a_0 = 3}$$

\subsubsection*{b)}
$$a_n = 2n$$
$$a_n - a_n = 2n - 2n + 2 $$
$$\boxed{a_n = a_{n-1}+2, a_0 = 0}$$

\subsubsection*{c)}
$$a_n = 2n+3$$
$$a_n - a_{n-1} = 2n+3 - 2n +2 - 3$$
$$\boxed{a_n = a_{n-1}+2, a_0 = 3}$$

\subsubsection*{d)}
$$a_n = 5^n$$
$$\frac{a_n}{a_{n-1}} = \frac{5^n}{5^{n-1}}$$
$$\boxed{a_n = 5a_{n-1}, a_0 = 1}$$

\subsubsection*{e)}
$$a_n - a_n{n-1} = n^2 - (n-1)^2$$
$$a_n = a_{n-1} +n^2 - (n-1)^2$$
$$\boxed{a_n = a_n +2n - 1, a_0 = 0}$$

\subsubsection*{f)}
$$a_n - a_{n-1} = n^2 + n - (n-1)^2 - (n-1)$$
$$\boxed{a_n = a_{n-1} +2n, a_0 = 0}$$

\subsubsection*{g)}
$$a_n + a_{n-1}= n+ (-1)^n + n - 1 +(-1)^{n-1}$$
$$\because 1^n = -(-1)^{n-1}$$
$$a_n + a_{n-1}= n + (-1)^n + n - 1 -(-1)^{n}$$
$$\boxed{a_n = -a_{n-1} +2n- 1, a_0 = 1}$$

\subsubsection*{h)}
$$\frac{a_n}{a_{n-1}} = \frac{n!}{(n-1)!}$$
$$\boxed{a_n = a_{n-1}n, a_0 = 1}$$

\subsection*{22}
\subsubsection*{a)}
$$\boxed{a_n = 1.05a_{n-1} + 1000, a_0 = 50000}$$


\subsubsection*{b)}
compute through iteration,

$$a_0 = 50000$$
$$a_1 = 53500$$
$$...$$
$$a_8 = 83418$$


\subsection*{32}
\subsubsection*{a)}
$$1 + (-1) = 0$$
$$1 + 1 = 2$$
Since there are 4 0's and 4 2's, the sum becomes
$$4*2 = \boxed{8}$$

\subsubsection*{b)}
$$\Sigma 3^j - \Sigma 2^j$$
clsoed form of both sums
$$\frac{3^9 - 1}{3-1} - \frac{2^9 - 1}{2-1}$$
$$9841 - 511 = \boxed{9330}$$

\subsubsection*{c)}
$$2 \Sigma 3^j + 3 \Sigma 2^j$$
closed form becomes 
$$2 \frac{3^9 - 1}{3-1} + 3 \frac{2^9 - 1}{2-1}$$
$$19682 + 1533 = \boxed{21215}$$

\subsubsection*{d)}
$$\Sigma 2^{j+1} - \Sigma 2^j$$
$$\frac{2^9 - 2}{2-1} - \frac{2^9 - 1}{2-1}$$
$$510 - 511 = \boxed{-1}$$

\section*{5.1}
\subsection*{8}
Prove:
$$2 \Sigma (-7)^n = \frac{1-(-7)^{n+1}}{4}$$
By induction, base case: n = 1
$$2 = \frac{1 + 7}{4} = 2$$
holds for k, now, n = k+1;
$$2 \Sigma (-7)^k + 2(-7)^{k+1} = \frac{1-(-7)^{k+2}}{4}$$
$$\frac{1-(-7)^{k+1}}{4} + 2(-7)^{k+1} = \frac{1-(-7)^{k+2}}{4}$$
$$\frac{1-(-7)^{k+1}(7)}{4} = \frac{1-(-7)^{k+2}}{4}$$
$$\frac{1-(-7)^{k+2}}{4} = \frac{1-(-7)^{k+2}}{4}$$
$$\boxed{\therefore 2 \Sigma (-7)^n = \frac{1-(-7)^{n+1}}{4}}$$

\subsection*{12}
By induction, n = 0;
$$1 = 1$$
holds for k, now, n = k+1;
$$\Sigma (-\frac{1}{2})^j + (-\frac{1}{2})^{k+1} = \frac{2^{k+2} + (-1)^{k+1}}{3(2)^{k+1}}$$
$$\frac{2}{2}*\frac{2^{k+1} + \frac{3}{3}*(-1)^{k}}{3(2)^{k}} + (-\frac{1}{2})^{k+1} = \frac{2^{k+2} + (-1)^{k+1}}{3(2)^{k+1}}$$
$$\frac{2^{k+2} + 2(-1)^{k} + 3(-1)^{k+1}}{3(2)^{k+1}} = \frac{2^{k+2} + (-1)^{k+1}}{3(2)^{k+1}}$$
$$\frac{2^{k+2} + (-1)^{k+1}}{3(2)^{k+1}} = \frac{2^{k+2} + (-1)^{k+1}}{3(2)^{k+1}}$$
$$\boxed{\therefore \Sigma (-\frac{1}{2})^j  = \frac{2^{n+1} + (-1)^{n}}{3(2)^{n}}}$$

\subsection*{14}
by induction, n = 1;
$$2=2$$
holds for k, now, n = k+1;
$$\Sigma j2^j + (k+1)2^{k+1}= k2^{k+2}+2$$
$$(k-1)2^{k+1} +2 + (k+1)2^{k+1}= k2^{k+2}+2$$
$$2^k(k-1+k+1) + 2 = k2^{k+2}+2$$
$$2k2^{k} + 2 = k2^{k+2}+2$$
$$k2^{k+2}+2 = k2^{k+2}+2$$
$$\boxed{\therefore \Sigma j2^j = (n-1)2^{n+1}+2}$$

\subsection*{18}
\subsubsection*{a)}
$$\boxed{P(2) \rightarrow 2! < 2^2}$$


\subsubsection*{b)}
$$\boxed{2 < 4}$$

\subsubsection*{c)}
$$\boxed{P(k) \rightarrow k! < k^k}$$

\subsubsection*{d)}
$$\boxed{(k+1)! < (k+1)^{k+1}}$$
\subsubsection*{e)}
$$(k+1)(k!) < (k+1)(k+1)^k$$
$$k! < k^k < (k+1)^k$$
$$\boxed{\therefore n! < n^n}$$

\subsubsection*{f)}
Since the inductive hypothesis extends the base case, 
if the hypothesis is true, then $P(k+1)$ is true since 
$k! < (k+1)^k$ as shown above. Finally, since it was shown 
that $P(n)$ holds for $k$ and $k+1$, $P(n)$ holds for all n within the specified domain.
\end{document}
