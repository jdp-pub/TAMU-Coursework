\documentclass[journal]{IEEEtran}

\usepackage{blindtext}
\usepackage{cite}
\usepackage{graphicx}
\usepackage{array}
\usepackage{color}
\usepackage{tabularx}
\usepackage{epsfig}
\usepackage{amsmath}
\usepackage{amssymb}
\usepackage{bm}
\usepackage{wasysym}
\usepackage{circuitikz}
\usepackage{float}
\usetikzlibrary{arrows,shapes,calc,positioning}

\newcommand{\myscope}[2]
{\draw[thick,rotate=#2] (#1) circle (12pt)
(#1) ++(-0.35,-0.1) --++ (0.3,0.3) --++ (0,-0.3) --++(0.3,0.3) --++(0,-0.3);}

\begin{document}

\title{CSCE 222 \\ Homework 7}

\author{Jacob~Purcell,~\IEEEmember{Texas~A\&M,~Student}}

\maketitle

\section*{8.2}
\subsection*{4}
\subsubsection*{D)}
$$x^n = 2x^{n-1} - x^{n-2}$$
$$x^2 = 2x - 1$$
$$x^2 - 2x + 1 = 0$$
$$(x - 1)(x - 1) = 0$$
$$x = 1,1$$
\\\\
$$a_n = c_1x^n + nc_2x^n$$
$$a_0 = 4 = c_1$$
$$a_1 = 1 = 4 + c_2$$
$$\boxed{a_n = 4 - 3n}$$

\subsubsection*{E)}
$$x^n = x^{n-2}$$
$$x^2 - 1 = 0$$
$$x = \pm 1$$
\\\\
$$a_n = c_1x^n + c_2x^n$$
$$a_0 = c_1 + c_2 = 5$$
$$a_1 = c_1 - c_2 = -1$$
$$2c_1 = 4, c_1 = 2, c_2 = 3$$
$$\boxed{a_n = 2 + 3(-1)^n}$$

\subsubsection*{G)}
$$x^{n+2} = -4x^{n+1} + 5x^n$$
$$x^2 + 4x - 5 = 0$$
$$x = -5, 1$$
\\\\
$$a_n = c_1(-5)^n + c_2$$
$$a_0 = c_1 + c_2 = 2$$
$$a_1 = -c_15 + c_2 = 8$$
$$-c_16 = 6, c_1 = -1, c_2 = 3$$
$$a_n = 3 - (-5)^n $$

\subsection*{8}
\subsubsection*{A)}
$$\boxed{L_n = \frac{L_{n-1} + L_{n-2}}{2}}$$

\subsubsection*{B)}
$$x^n = \frac{x^{n-1} + x^{n-2}}{2}$$
$$2x^2 - x^{1} - 1 = 0$$
$$(2x + 1)(x - 1) = 0$$
\\\\
$$L_n = c_1(-\frac{1}{2})^n + c_2$$
$$L_1 = 100,000 = c_1(-\frac{1}{2}) + c_2$$
$$L_2 = 300,000 = c_1(\frac{1}{4}) + c_2$$
$$200,000 = c_1(\frac{3}{4})$$
$$c_1 = 266,666 \frac{2}{3}, c_2 = 233,333 \frac{1}{3}$$
$$\boxed{L_n = 266,666 \frac{2}{3}(-\frac{1}{2})^n + 233,333 \frac{1}{3}}$$

\subsection*{12}
$$x^n = 2x^{n-1} + x^{n-2} - 2 x^{n-3}$$
$$x^3 = 2x^2 + x - 2$$
Using Mathematica,
$$x = -1, 1, 2$$
\\\\
$$a_n = c_1(-1)^n + c_2 + c_3 2^n$$
$$a_0 =3= c_1 + c_2 + c_3$$
$$a_1 =6= c_1(-1) + c_2 + c_3 2$$
$$a_2 =0= c_1 + c_2 + c_3 4$$
\\\\
$$9=  2 c_2 + 3 c_3$$
$$6 = 2c_2 + 6c_3$$
$$3 = -3c_2$$
$$c_3 = -1, c_2 = 6, c_1 = -2$$
$$\boxed{a_n = -2(-1)^n + 6 - 2^n}$$

\subsection*{14}
$$x^n = 5 x^{n-2} - 4 x^{n-4}$$
$$x^4 - 5 x^2 + 4 = 0$$
$$(x^2 - 4)(x^2 - 1) = 0$$
$$x = 1, -1, 2, -2$$
\\\\
$$a_n = c_1 + c_2 (-1)^n + c_3 2^n + c_4 (-2)^n$$
$$a_0 =3= c_1 + c_2 + c_3 + c_4$$
$$a_1 =2= c_1 - c_2 + c_3 2 - c_4 2$$
$$a_2 =6= c_1 + c_2 + c_3 4 + c_4 4$$
$$a_3 =8= c_1 - c_2 + c_3 8 - c_4 8$$
\\\\
$$5 = 2c_1 + 3 c_3 - c_4$$
$$14 = 2c_1 + 12 c_3 - 4 c_4$$
$$8 = 2c_1 + 6 c_3 + 2 c_4$$
\\\\
$$9 = 9c_3 - 3c_4$$
$$3 = 3c_3 + 3c_4$$
\\\\
$$12 = 12c_3$$
$$c_3 = 1, c_4 = 0, c_1 = 1, c_2 = 1$$
$$\boxed{a_n = 1 + (-1)^n + 2^n}$$


\subsection*{24}
\subsubsection*{A)}
First, adjust closed form for $a_{n-1}$,
$$a_{n-1} = (n-1)2^{n-1}$$
Plug into recurrence relation,
$$a_n = 2((n-1)2^{n-1}) + 2^n$$
Simplify,
$$a_n = (n-1)2^{n} + 2^n$$
$$a_n = n2^{n}$$
Which is the original expression. 

$$\boxed{\therefore n2^{n} = 2a_{n-1} + 2^n}$$

\subsubsection*{B)}
$$x^n = 2x^{n-1} + F(n)$$
$$x = 2 + F(n)$$
$$x = 2, F(n) = 2^n$$
$$a_n = \alpha 2^n + \alpha ^(p)$$
$$\alpha ^(p) = n^m p_0 s^n$$
$$\alpha ^(p) = n(n p_1 + p_0) 2^n$$
$$\boxed{a_n = \alpha 2^n + n(n p_1 + p_0) 2^n}$$

\section*{9.1}
\subsection*{6}
\subsubsection*{A)}
$$x + y = 0$$

Reflexive:
$$x = 1$$
$$xRx = 2, 2 \neq 1, \therefore \boxed{Not~Reflexive}$$

Symmetric:
Because addition is commutative, $x+y = y+x$

$$\therefore \boxed{Symmetric}$$

Antisymmetric:
$$x = -y, y = -x, x = y~iff~x,y=0$$
$$\therefore \boxed{Not~Antisymmetric}$$

Transitive:
$$x = 1, y = -1, z = 1$$
$$xRy \in R,  yRz \in R, xRz \notin R$$
$$\therefore \boxed{Not~Transitive}$$


\subsubsection*{C)}
$$R: x-y \in \mathbb{Q}, x, y \in \mathbb{R}$$

Reflexive:
$$x = x$$
$$xRx = 0~\forall~x$$
$$\because 0 \in \mathbb{Q}, \boxed{Reflexive}$$

Symmetric:
Due to the properties of subtraction, if $xRy$ is on 
$\mathbb{Q}$, so is $yRx$.
$$\therefore \boxed{Symmetric}$$

Antisymmetric:
$$x = -y \in \mathbb{Q}, y = -x \in \mathbb{Q}, x = y~iff~x,y=0$$
$$\therefore \boxed{Not~Antisymmetric}$$

Transitive:
$$x \in Q, y \in Q, z \in Q$$
$$xRy \in Q,  yRz \in Q, xRz \in Q$$
$$\therefore \boxed{Transitive}$$

\subsubsection*{D)}
Reflexive:
$$xRx = 2x, 2x \neq x$$
$$\therefore \boxed{Not~Reflexive}$$

Symmetric:
$$xRy \in \mathbb{R}, yRx \in \mathbb{R}$$
$$\therefore \boxed{Symmetric}$$

Antisymmetric:
$$x = 2y, y = 2x, x = y~iff~x,y=0$$
$$\therefore \boxed{Not~Antisymmetric}$$

Transitive:
$$x = 4, y = 2, z = 1$$
$$xRy \in R,  yRz \in R, xRz \notin R$$
$$\therefore \boxed{Not~Transitive}$$

\subsubsection*{E)}
Reflexive:
$$xx \ge 0 \because x^2 \ge 0 \forall x \in \mathbb{R}, x = x$$
$$\therefore \boxed{Reflexive}$$

Symmetric:
$$xRy \in \mathbb{R}, yRx \in \mathbb{R}$$
$$\therefore \boxed{Symmetric}$$

Antisymmetric:
$$xy \ge 0 \rightarrow yx \ge 0$$
$$\therefore \boxed{Antisymmetric}$$

Transitive:
$$x = -1, y = 0, z = 1$$
$$xy \in R,  yz \in R, xz \notin R$$
$$\therefore \boxed{Not~Transitive}$$


\end{document}
