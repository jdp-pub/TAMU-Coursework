\documentclass[journal]{IEEEtran}

\usepackage{blindtext}
\usepackage{algorithm}
\usepackage{algorithmic}
\usepackage{cite}
\usepackage{graphicx}
\usepackage{array}
\usepackage{color}
\usepackage{tabularx}
\usepackage{epsfig}
\usepackage{amsmath}
\usepackage{amssymb}
\usepackage{bm}
\usepackage{wasysym}
\usepackage{circuitikz}
\usepackage{float}
\usetikzlibrary{arrows,shapes,calc,positioning}

\newcommand{\myscope}[2]
{\draw[thick,rotate=#2] (#1) circle (12pt)
(#1) ++(-0.35,-0.1) --++ (0.3,0.3) --++ (0,-0.3) --++(0.3,0.3) --++(0,-0.3);}

\begin{document}

\title{CSCE 221 \\ Problem Set 8}

\author{Jacob~Purcell,~\IEEEmember{Texas~A\&M,~Student}}

\maketitle
\section{}
For O(N), a linear search will be sufficient. Have an iterator store its starting pointer, 
then follow each subsequent pointer until it finds a match. This will iterate through every 
value and requires O(N) time to compute. With O(1) extra space, another iterator may be added but 
traverses 2 nodes instead of one, eventually the fast iterator will catch up to the slow iterator. 
While the slow iterator is still testing against whether it makes a circle, the fast iterator can 
test whether it has caught up to the slow iterator. 

\begin{algorithm}
\caption{Check for Circular List}
\begin{algorithmic}
	\REQUIRE $N~is~an~element~of~\mathbb{N}$
	\STATE $slow = head, fast = head$
	\WHILE{$slow != nullptr \wedge fast != nullptr$}
	\STATE $slow \rightarrow next(O(N))$
	\STATE $fast \rightarrow next \rightarrow next (one~extra~calculation)$
	\IF{$slow == head || fast == slow$}
	\STATE $Return~true$
	\ENDIF
	\ENDWHILE
	\RETURN $false$
\end{algorithmic}
\end{algorithm}


\section{}
\subsection{}
\begin{algorithm}
\caption{Selfadjusting List}
\begin{algorithmic}
	\REQUIRE $Array$
	\STATE $temporary~storage = requested~value~O(1)$
	\STATE $iterator = requested~index~O(N)$
	\WHILE {$iterator~is~greater~than~zero$}
	\STATE $copy~previous~array~element~to~current$
	\STATE $~~~~location~O(1)$
	\STATE $decrement~iterator~O(1)$ 
	\STATE $(removes~requested~value~from~array~and~$ \\ $~~~~creates~a~free~slot~at~front)$
	\ENDWHILE $O(N)$
	\STATE $assign~temporary~storage~to~Array~position~zero~O(1)$
	\RETURN $temporary~storage$
\end{algorithmic}
\end{algorithm}

~\\\\
~\\\\
~\\\\

\subsection{}
\begin{algorithm}
\caption{Selfadjusting List}
\begin{algorithmic}
	\REQUIRE $Doubly~Linked~List$
	\STATE $cursor = head~O(1)$
	\WHILE {$cursor~is~not~nullptr~and~cursor~data~is~not$ \\ $requested~data$}
	\STATE $cursor \rightarrow next~O(1)$ \\ $(finds~desired~value~)~O(N)$
	\ENDWHILE 
	\STATE $cursor~data = temporary~storage~O(1)$
	\STATE $cursor~previous~next \rightarrow cursor~next~O(1)$
	\STATE $cursor~next~previous~\rightarrow~cursor~previous~O(1)$ \\ $(links~list~around~the~desired~ptr)$
	\STATE $cursor~next \rightarrow head~O(1)$
	\STATE $cursor~previous \rightarrow nullptr~O(1)$
	\STATE $head = cursor~O(1)$ \\ $(reorder~list~so~last~accessed~ptr~is~at~front)$
	\RETURN $temprorary~storage$
\end{algorithmic}
\end{algorithm}


\subsection{}
The time complexity for both implementations is the same at the asymptote (O(N)), however, the array implementation 
would be noticably slower for small datasets with a time complexity of O(2N). For each, the time complexity will 
drop to O(1) if the same element is requested more than once in a row.

\subsection{}
Every element has a certain probability ($p_i$) of being accessed. After being accessed, it is $100\%$ likely that 
the value will be at the front of the list the next iteration with a probability $p_i$ of being accessed again.
 Introducing $\hat{e_k}$, the index of the element, the last accessed element is always $\hat{e_0}$. After $N$ accesses, 
 the probability of being accessed $n_i$ times is $Np_i$. $k$ therefore holds the form $\frac{\hat{i}}{n_i}$, where the index
 of the element is inversely related to it's probability, $\hat{i}$ is a normalized position and holds the numerical 
 value of 1. Finally, this means that the chance of finding any specific value in any position is ordered 
 by $\boxed{\Sigma_{i = 1}^{Elements} n_i \hat{e_k}}$.





\end{document}
